\documentclass{article}
\usepackage{fullpage,amsmath,amsthm,graphicx,enumitem}
\usepackage{multicol}
\usepackage{booktabs}
\usepackage{blkarray}

\usepackage{tikz}

\theoremstyle{definition}
\newtheorem{thm}{Theorem}
\newtheorem{question}[thm]{Question}
\newenvironment{solution}{\noindent\textit{Solution:}}{}

\title{ASEN 5519-002 Decision Making under Uncertainty\\
       Quiz 2: Reinforcement Learning and POMDPs}

\date{\small Show all work and justify and box answers.\\
You may consult any source, but you may NOT communicate with any person except the instructor.\\
If you run into a problem that you don't know how to solve, \emph{skip it} and come back later to maximize your score.
}

\begin{document}
\maketitle

\begin{question} (20 pts)
    You enter a casino and there are two slot machines, $A$ and $B$, that pay either \$0 or \$1 per play, each with potentially different winning probabilities. You have been playing for a few rounds and keeping track of the outcomes of each attempt and want to use UCB1 exploration with exploration parameter $c=\$1$ to select your actions.
    \begin{enumerate}[label=\alph*)]
        \item If you have played $A$ 10 times, receiving winnings of \$8, and $B$ 4 times, winning \$3. Which machine should you play next according to UCB1 exploration? Justify your answer quantitatively.
        \item If you have played $A$ 10 times, winning \$9, and $B$ 10 times, winning \$8. Which machine should you play next according to UCB1 exploration? Justify your answer quantitatively.
        \item In which of the above situations did UCB1 exploration select the same action as a greedy policy? Justify your answer.
    \end{enumerate}
\end{question}

\begin{question} (30 pts)
    Consider a 2-state, 2 action POMDP with $\mathcal{S} = \{1,2\}$ and $\mathcal{A} = \{0,1\}$. State 2 is terminal and the discount factor is $\gamma=0.9$. Suppose that you are performing reinforcement learning, and you observe an episode that takes the following 3-step trajectory:
    $$(s=1, a=0, r=1, s'=1)$$
    $$(s=1, a=0, r=1, s'=1)$$
    $$(s=1, a=1, r=1, s'=2)$$

    \begin{enumerate}[label=\alph*)]
        \item (6 pts) Suppose you are using \textbf{maximum likelihood model-based reinforcement learning (MLMBRL)}. After observing the trajectory above, what are the maximum likelihood transition probabilities for action $a=0$?
        % \item Suppose that you are using the \textbf{SARSA} algorithm with learning rate $\alpha=0.1$ and all $Q$ values starting at 0 before the episode. What are the $Q$ value estimates after the episode?
        \item (18 pts) Suppose that you are using the \textbf{Q-learning} algorithm with learning rate $\alpha=0.1$ and all $Q$ values starting at 0 before the episode. What are the $Q$ value estimates after the episode?
        \item (6 pts) Suppose that you are using \textbf{policy gradient} with a policy parameterized with $\theta = [\theta_1, \theta_2]$ defined as
            $$\pi_\theta (a=1 | s=i) = \left[\theta_i\right]_0^1$$
            where $\left[x\right]_a^b = \text{clamp}(x, a, b) = \min(b, \max(a, x))$. That is, $\theta_i$ is probability of taking action 1 in state $i$.  If the parameter values are $\theta = [0.5, 0.5]$, what is the policy gradient estimate calculated from the trajectory above? Do not use baseline subtraction. You may find the following derivatives useful: $\frac{\partial}{\partial \theta_i} \log(\theta_i) = \frac{1}{\theta_i}$, $\frac{\partial}{\partial \theta_i} \log(1-\theta_i) = -\frac{1}{1-\theta_i}$.
    \end{enumerate}
\end{question}

\begin{question} (50 pts)
    You have been tasked with preventing poaching at a large national park. This problem can be formulated as a POMDP with two states: either there are poachers ($P$), or the park is clear ($C$). There are three actions, fly a UAV over the park ($U$), send in rangers in jeeps ($J$), or wait ($W$). At each step, you receive an observation of the state ($\mathcal{O} = \{P, C\}$). If the action is wait, both observations are equally likely. If the UAV or jeeps are employed, the observation is always accurate. Action $J$ eliminates poachers immediately, otherwise their presence remains unchanged. Action $J$ always results in a cost of 5 regardless of the state; waiting ($W$) has no cost if the state is $C$, and a cost of 10 if the state is $P$; the UAV ($U$) has a cost of 1 if the state is $C$ and 11 if the state is $P$.
    In summary,
    \begin{align}
        \mathcal{S} &= \mathcal{O} = \{P, C\} \quad \\
        \mathcal{A} &= \{U, J, W\} \\
        \mathcal{R}(s, a) &= \begin{cases}
            -5 \text{ if } a = J \\
            -10 \text{ if } a = W \text{ and } s = P \\
            -1 \text{ if } a = U \text{ and } s = C \\
            -11 \text{ if } a = U \text{ and } s = P \\
            0 \text{ otherwise}
        \end{cases}\\
        \mathcal{T}(s' \mid s, a) &= \begin{cases}
            1 \text{ if } a \in \{U, W\} \text{ and } s' = s\\
            1 \text{ if } a = J \text{ and } s' = C\\
            0 \text{ otherwise}
        \end{cases}\\
        \mathcal{Z}(o \mid a, s') &= \begin{cases}
            1 \text{ if } a \in \{U, J\} \text{ and } o = s' \\
            0 \text{ if } a \in \{U, J\} \text{ and } o \neq s' \\
            0.5 \text{ if } a = W
        \end{cases} \\
        \gamma &= 1
    \end{align}
    ($\gamma=1$ means this problem is ill-defined for an infinite horizon, but we only consider finite-horizon plans).
  
    \begin{enumerate}[label=\alph*)]
        \item Calculate and write out \textbf{one step} alpha vectors for each action.
        \item Draw and label the \textbf{one step} alpha vectors in the manner done in class.
        \item According to the policy defined by the one-step alpha vectors above, under what circumstances would you take the $U$ action? Why?
        \item Suppose you use a certainty-equivalent approach that takes the best action for the most likely state. Which action would this CE approach \emph{avoid} in this POMDP\footnote{For this part only, you can assume $\gamma<1$}? Why?
        \item Draw diagrams\footnote{similar to Figure 20.1 in the book} for the following \textbf{two step} conditional plans:
            \begin{enumerate}
                \item Always wait ($W$)
                \item Always send in rangers on jeeps ($J$)
                \item Fly the UAV ($U$), then wait ($W$) if the observation is clear ($C$) or send in jeeps ($J$) if poachers are detected ($P$)
            \end{enumerate}
        \item Calculate and write out the alpha vectors for the \textbf{two step} conditional plans above.
        \item Draw and label the \textbf{two step} alpha vectors for the conditional plans above in the manner done in class.
        \item In a policy defined by the \textbf{two step} plans above, what action would be selected if the belief is uniform (i.e. $b(P) = 0.5$)?
    \end{enumerate}
\end{question}

\end{document}
